\sectioncentered*{Введение}
\addcontentsline{toc}{section}{Введение}
\label{sec:intro}
Постиндустриальное общество как этап развития цивилизации основано на информации как на основном производственном ресурсе. Количественные и качественные характеристики информации являются существенными параметрами многих процессов, происходящих в мире.
Соответственно пониманию информации как ресурса существует ее представление как собственности. Как результат интеллектуальной деятельности, данные могут быть присвоены и защищены. Однако законодательного установления собственности на те или иные информационные разработки иногда бывает недостаточно для их защиты от злоумышленников. В соответствии с потребностями нашего времени, проводится всё больше исследований на тему защиты продуктов интеллектуальной деятельности различными криптографическими методами.
Относительно такой разновидности продукта, как программное обеспечение (программные коды), в качестве защиты может применяться обфускация – методология трансформации кода с сохранением его функциональности. Трансформация представляет собой изменение характера, вида и порядка инструкций. Её целью является усложнение анализа, понимания и, как следствие, нежелательного использования продукта интеллектуальной деятельности посторонними лицами.
Идеальной реализацией обфускатора считается средство, способное преобразовать код в такую последовательность инструкций, которая при использовании демонстрировала бы только целевой результат, но не способ достижения цели. То есть позволяла проследить факт преобразования входных данных в конкретные выходные без отображения того, как именно это было сделано.
Существующие на данный момент обфускаторы в той или иной мере приближены к «лучшей» реализации, но пока не достигают её.
Данный дипломный проект направлен на продолжение исследований в вышеописанной области и создание программного средства, способного производить лексическую и функциональную обфускацию проектных описаний цифровых устройств.