\sectioncentered*{Заключение}
\addcontentsline{toc}{section}{Заключение}

В данном дипломном проекте был рассмотрен вопрос лексического и синтаксического анализа языка VHDL, также различные методики запутывания, как исходного кода, так и результата синтеза этого кода. В рамках дипломного проекта была разработана библиотека кода для анализа и обфускации исходных кодов на языке VHDL.
В разработанном проекте был использован генератор парсеров YACC для создания анализатора языка, а так же произвольный набор правил, построенный на основе BNF-грамматики языка и обфускатор, представленный наборов классов.

В целом получены хорошие результаты обработки и обфускации на ряде исходных кодов. Время работы приложения является линейным и зависит от объёма входных данных. Результаты работы реализованных в проекте функций замены различных типов литералов в большинстве случаев превосходят по качеству функциональность уже существующих открытых аналогов.

В итоге получилось раскрыть тему дипломного проекта и создать в его рамках программное обеспечение.
Но за рамками рассматриваемой темы осталось еще много других алгоритмов синтаксического и лексического анализа, а также различных приёмов обфускации.

В дальнейшем планируется развивать и довести существующее ПО до полноценной библиотеки, способной решать более широкий класс задач, возникающих в области защиты интеллектуальной собственности и запутывания кода.