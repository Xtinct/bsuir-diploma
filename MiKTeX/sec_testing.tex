\section{Тестирование приложения}
\label{sec:testing}

Для оценки правильности работы программного средства было проведено тестирование. Тест-кейсы для функционального требования <<Взаимодействие с пользователем>> представлены в таблице \ref{sec:testing:interaction_cases}



\begin{longtable}{| >{\centering}p{0.3\textwidth} | >{\centering}p{0.3\textwidth} | >{\centering\arraybackslash}p{0.3\textwidth}|}
  \caption{Тестирование взаимодействия с пользователем}
  \label{sec:testing:interaction_cases} \tabularnewline

  \hline
       Название тест-кейса и его описание & Ожидаемый результат  & Полученный результат \\
   \hline
   1 & 2 & 3 \\
   \hline
   Запуск программы без аргументов
   \begin{enumerate}
   \item Ввести имя исполняемого модуля без аргументов
   \item нажать клавишу Ввода
   \end{enumerate}
   &
   \begin{enumerate}
   \item Имя исполняемого модуля отображается в консоли
   \item Отображается справочная информация об использовании приложения
   \end{enumerate}
   &
   \begin{enumerate}
   \item Имя исполняемого модуля отображается в консоли
   \item Отображается справочная информация об использовании приложения
   \end{enumerate} \\
   \hline

   Запуск программы с неверными аргументами
   \begin{enumerate}
   \item Ввести имя исполняемого модуля с неверными аргументами
   \item нажать клавишу Ввода
   \end{enumerate}
   &
   \begin{enumerate}
   \item Имя исполняемого модуля и аргументы отображаются в консоли
   \item Отображается ошибка о вводе неправильного аргумента(аргументов)
   \end{enumerate}
   &
   \begin{enumerate}
   \item Имя исполняемого модуля и аргументы отображаются в консоли
   \item Отображается ошибка о вводе неправильного аргумента(аргументов)
   \end{enumerate} \\
   \pagebreak

   \caption*{Продолжение таблицы~\ref{sec:testing:interaction_cases}} \\
   \hline
   1 & 2 & 3 \\
   \hline
   Запуск программы с переданным путём до существующего файла
   \begin{enumerate}
   \item Ввести имя исполняемого модуля и передать путь до существующего файла как аргумент
   \item нажать клавишу Ввода
   \end{enumerate}
   &
   \begin{enumerate}
   \item Имя исполняемого модуля с путём до существующего файла отображаются в консоли
   \item Программа выводит обфусцированный код в консоль
   \end{enumerate}
   &
   \begin{enumerate}
   \item Имя исполняемого модуля с путём до существующего файла отображаются в консоли
   \item Программа выводит обфусцированный код в консоль
   \end{enumerate} \\
   \hline
   Запуск программы с несуществующим файлой
   \begin{enumerate}
   \item Ввести имя исполняемого модуля и передать путь до несуществующего файла как аргумент
   \item нажать клавишу Ввода
   \end{enumerate}
   &
   \begin{enumerate}
   \item Имя исполняемого модуля с путём до несуществующего файла отображаются в консоли
   \item Программа выводит ошибку о том, что файл не может быть найден
   \end{enumerate}
   &
   \begin{enumerate}
   \item Имя исполняемого модуля с путём до существующего файла отображаются в консоли
   \item Программа выводит ошибку о том, что файл не может быть найден
   \end{enumerate}\\
   \pagebreak
   \caption*{Продолжение таблицы~\ref{sec:testing:interaction_cases}} \\
   \hline
   1 & 2 & 3 \\
   \hline

   Запуск программы с входным и выходным файлом
   \begin{enumerate}
   \item Ввести имя исполняемого модуля и передать путь до существующего файла и файла вывода как аргументы
   \item нажать клавишу Ввода
   \item открыть файл вывода
   \end{enumerate}
   &
   \begin{enumerate}
   \item Имя исполняемого модуля с путём до существующего файла и файла вывода отображаются в консоли
   \item Программа выполняет работу
   \item Конечный файл содержит результаты работы программы
   \end{enumerate}
   &
   \begin{enumerate}
   \item Имя исполняемого модуля с путём до существующего файла и файла вывода отображаются в консоли
   \item Программа выполняет работу
   \item Конечный файл содержит результаты работы программы
   \end{enumerate} \\
   \hline
   Запуск программы с входным файлом и флагом --lexical-only
   \begin{enumerate}
   \item Ввести имя исполняемого модуля и передать путь до существующего файла с аргументов --lexical-only
   \item нажать клавишу Ввода
   \end{enumerate}
   &
   \begin{enumerate}
   \item Имя исполняемого модуля с путём до существующего файла и аргумент отображаются в консоли
   \item Программа не содержит функциональной обфускации
   \end{enumerate}
   &
   \begin{enumerate}
   \item Имя исполняемого модуля с путём до существующего файла и аргумент отображаются в консоли
   \item Программа не содержит функциональной обфускации
   \end{enumerate} \\
    \pagebreak
   \caption*{Продолжение таблицы~\ref{sec:testing:interaction_cases}} \\
   \hline
   1 & 2 & 3 \\
   \hline
   Запуск программы с входным файлом и флагом --functional-only
   \begin{enumerate}
   \item Ввести имя исполняемого модуля и передать путь до существующего файла с аргументов --functional-only
   \item нажать клавишу Ввода
   \end{enumerate}
   &
   \begin{enumerate}
   \item Имя исполняемого модуля с путём до существующего файла и аргумент отображаются в консоли
   \item Программа содержит функциональную обфускацию, но не содержит лексической
   \end{enumerate}
   &
   \begin{enumerate}
   \item Имя исполняемого модуля с путём до существующего файла и аргумент отображаются в консоли
   \item Программа содержит функциональную обфускацию, но не содержит лексической
   \end{enumerate} \\
   \hline
\end{longtable}
   Тест-кейсы для функционального требования <<Анализирование входных файлов>> представлены в таблице \ref{sec:testing:analyzing_cases}:

\begin{longtable}{| >{\centering}m{0.3\textwidth}
                  | >{\centering}m{0.3\textwidth}
                  | >{\centering\arraybackslash}m{0.3\textwidth}|}
  \caption{Тестирование взаимодействия с пользователем}
  \label{sec:testing:analyzing_cases} \tabularnewline

  \hline
    Название тест-кейса и его описание & Ожидаемый результат & Полученный результат
    \tabularnewline
   \hline
   1 & 2 & 3 \\
   \hline
   Запуск программы с действительным VHDL кодом
   \begin{enumerate}
   \item Ввести имя исполняемого модуля с входным файлом, являющимся правильным VHDL-кодом
   \item нажать клавишу Ввода
   \end{enumerate}
   &
   \begin{enumerate}
   \item Имя исполняемого модуля и путь до файла отображается в консоли
   \item Генерируется правильное абстрактное синтаксическое дерево
   \end{enumerate}
   &
   \begin{enumerate}
   \item Имя исполняемого модуля и путь до файла отображается в консоли
   \item Генерируется правильное абстрактное синтаксическое дерево
   \end{enumerate} \\

  \pagebreak
  \caption*{Продолжение таблицы~\ref{sec:testing:analyzing_cases}} \\
   \hline
   1 & 2 & 3 \\
   \hline
   Запуск программы с недействительным VHDL кодом
   \begin{enumerate}
   \item Ввести имя исполняемого модуля с входным файлом, являющимся неправильным VHDL-кодом
   \item нажать клавишу Ввода
   \end{enumerate}
   &
   \begin{enumerate}
   \item Имя исполняемого модуля и путь до файла отображается в консоли
   \item Выводится сообщения об ошибке анализа
   \end{enumerate}
   &
   \begin{enumerate}
   \item Имя исполняемого модуля и путь до файла отображается в консоли
   \item Выводится сообщения об ошибке анализа
   \end{enumerate} \\
   \hline
\end{longtable}
Таким образом, результат тестирования подтверждает, что программное средство лексической и функциональной обфускации проектных описаний цифровых устройств функционирует в полном соответствии со спецификацией требований.
  % \begin{longtable}{| >{\raggedright}m{0.3\textwidth}
  %                 | >{\centering}m{0.3\textwidth}
  %                 | >{\centering}m{0.3\textwidth}|}
  % \caption{Тестирование анализирования входных файлов}
  % \label{sec:testing:analyzing_cases} \tabularnewline

  % \hline
  %      Название тест-кейса и его описание & Ожидаемый результат & Полученный результат
  %    \tabularnewline
  %  \hline
  %  1 & 2 & 3 \\
  %  \hline
  %  Запуск программы без аргументов
  %  \begin{enumerate}
  %  \item Ввести имя исполняемого модуля без аргументов
  %  \item нажать клавишу Ввода
  %  \end{enumerate}
  %  &
  %  \begin{enumerate}
  %  \item Имя исполняемого модуля отображается в консоли
  %  \item Отображается справочная информация об использовании приложения
  %  \end{enumerate}
  %  &
  %  \begin{enumerate}
  %  \item Имя исполняемого модуля отображается в консоли
  %  \item Отображается справочная информация об использовании приложения
  %  \end{enumerate} \\
  %  \hline
   %-----------------------------------------
  % \caption*{Продолжение таблицы~\ref{table:econ:calculation_cost_and_price}}
  % \hline
  %   {\begin{center}
  %      Наименование статей
  %   \end{center} } & \mbox{Норматив,} \% & Методика расчета & \mbox{Значение,} руб. \\
  %  \hline
   % Прогнозируемая цена без налогов
   % &
   % & $ \text{Ц}_{\text{п}} = \text{С}_{\text{п}} + \text{П}_{\text{с}}$
   % & \num{\estimatedPrice} \\
   % \hline
   % Отчисления и налоги в местный и республиканский бюджеты
   % & $ \text{Н}_{\text{мр}} = \num{\localRepubTaxNormative} $
   % & $ \text{О}_{\text{мр}} = { \text{Ц}_{\text{п}} \cdot \text{Н}_{\text{мр}} } / { \num{100} - \text{Н}_{\text{мр}} } $
   % & \num{\localRepubTax} \\
   % \hline
   % Налог на добавленную стоимость
   % & $ \text{Н}_{\text{дс}} = \num{\ndsNormative} $
   % & $ \text{НДС}_{\text{}} = { (\text{Ц}_{\text{п}} + \text{О}_{\text{мр}}) \cdot \text{Н}_{\text{дс}} } / \num{100} $
   % & \num{\nds} \\
   % \hline
   % Прогнозируемая отпускная цена
   % &
   % & $ \text{Ц}_{\text{о}} = \text{Ц}_{\text{п}} + \text{О}_{\text{мр}} + \text{НДС} $
   % & \num{\sellingPrice} \\
   % \hline
